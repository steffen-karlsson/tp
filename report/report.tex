\documentclass[10pt]{IEEEtran}
\pdfoutput=1

\usepackage{graphicx}
\usepackage{hyperref}
\usepackage[utf8]{inputenc}
\usepackage{listings}
\usepackage[table]{xcolor}
\usepackage{pdfpages}

\hypersetup{colorlinks=true,citecolor=[rgb]{0,0.4,0}}


\title{Improving Trustpilot's rating system with datamining and machine learning}
\author{Steffen Karlsson \& Rune Thor Mårtensson}

\begin{document}
\maketitle

\begin{abstract}
The aim of this project is to improve the rating system of Trustpilot.dk, by utilizing data mining and machine learning. The way the improvement is supposed to work, is that a machine will be able to read reviews, get an understanding of what the customer is trying to convey about the company, and use the information to make a differentiated rating.
\end{abstract}

\section{Introduction}
When reading reviews at Trustpilot, you often see reviews which are very one-sided, only talking about how good a store is at handling delivery, how good their prices are or how good their presales support is. The problem with Trustpilot's rating system is, that it is merely the average of all ratings for the company. As users do not touch on all subjects the system is inherintly flawed, as a user who only touches one subject is rated the same as a user who touches all.

An example would be a customer writing a review, which only has information about how quick the company is to deliver a product. It is reasonable to presume, that the rating does not tell anything, about how good the company is at handling RMAs (Return Merchandize Authorization). 

By reading reviews and classifying the rating as something that is only related to delivery time, it is possible to make a differentiated rating, which this project achieves by using topic classification and sentiment analysis.

\section{Methods}
As the purpose of this project is to improve the rating system of a website, it makes sense to extend its functionality. This is achieved by using a \texttt{Greasemonkey}\cite{GreaseMonkey} script and an accompanying website which the script can acquire data from using \href{http://en.wikipedia.org/wiki/Ajax_(programming)}{\texttt{Ajax}}.
The system requires a web server, for this we use \href{http://httpd.apache.org/}{\texttt{Apache HTTP Server}}, as it is relatively easy to setup when using Linux. As the system needs to store data such as the differentiated ratings, a database is necessary, for which we use \href{http://www.mysql.com/}{\texttt{MySQL}}.

As the project is part of the course \href{http://www.kurser.dtu.dk/02819.aspx?menulanguage=en-GB}{02819 Data Mining using Python}, the project is coded in Python and uses language specific libraries.
\\~
\\~
The system acquires data, by scraping Trustpilot's website, for which a framework was created, that extends the functionality of Pythons built-in \href{http://docs.python.org/2/library/htmlparser.html}{\texttt{HTMLParser}}. By using this framework, it is possible to define a parser using a simple syntax - that defines what elements from a HTML page that should be acquired - which improves maintainability, while keeping the performance advantages of using an event driven parser such as HTMLParser. \newline As an event driven parser only iterates over the data once, it provide a performance advantage over other parsers such as \href{http://www.crummy.com/software/BeautifulSoup/}{\texttt{BeautifulSoup}}, which - depending on usage - can iterate over the same data more than once.
\\~
\\~
To provide an interface to the database, the ORM (Object Relational Mapper) \href{http://peewee.readthedocs.org/en/latest/}{\texttt{peewee}} is used. Peewee provides an easy to use interface between Python and a SQL database, in addition to this it provides protection against the commonly known vulnerability \href{https://www.owasp.org/index.php/Top_10_2013-A1-Injection}{\texttt{SQL injection}}.
\\~
\\~
When processing the acquired data, the system has two major components; the first is the multi topic classifier, which helps identifying topics within a review, the second component does sentiment analysis, which provides a positive or negative score for a review based on affective norms from the danish version of the word list AFINN-111\cite{IMM2011-06010}. The multi topic classifier has the ability to train an infinite number of the implementation of \href{http://nltk.org/api/nltk.classify.html#module-nltk.classify.naivebayes}{\texttt{Naïve Bayes classifers}} from nltk, which the system uses to classify review text in different categories.
\\~
\\~
To allow administrators to use the system, it has a number of administrative scripts. Which makes it possible to create jobs that scrape reviews for a company, scrape companies from a category and compute ratings based on the scraped reviews. Jobs are executed using \href{http://pubs.opengroup.org/onlinepubs/009696699/utilities/crontab.html}{\texttt{cron}} which is a time based job scheduler found in Linux and Unix.
\\~
\\~
In addition to these components, the system has a simple website, which holds the Greasemonkey script, information about the project and a ink to the structured documentation generated with \href{http://sphinx-doc.org/}{\texttt{Sphinx}}. The website uses \href{http://getbootstrap.com/}{\texttt{Bootstrap}} with a slightly modified template from \href{http://bootswatch.com/flatly/}{\texttt{bootswatch.com}}, which makes it easy to design a website.
\\~
\\~
The system is hosted on a micro instance from Amazon EC2, which runs the Debian GNU/Linux distribution. The system is hosted offsite, because it neccesary to keep the system running 24/7 while acquiring data and serving results to users.

\section{Results}


\subsection{Code checking}


\subsection{Testing}


\subsection{Profiling}


\section{Discussion}


\section{Conclusion}

\bibliographystyle{IEEEtran}
\bibliography{References}

\end{document}
