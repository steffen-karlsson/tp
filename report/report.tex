\documentclass[10pt]{IEEEtran}
\pdfoutput=1

\usepackage{graphicx}
\usepackage{hyperref}
\usepackage[utf8]{inputenc}
\usepackage{listings}
\usepackage[table]{xcolor}
\usepackage{pdfpages}

\hypersetup{colorlinks=true,citecolor=[rgb]{0,0.4,0}}


\title{Improving Trustpilot's rating system with datamining and machine learning}
\author{Steffen Karlsson \& Rune Thor Mårtensson}

\begin{document}
\maketitle

\begin{abstract}
The aim of this project is to improve the rating system of Trustpilot.dk, by utilizing data mining and machine learning. The way the improvement is supposed to work, is that a machine will be able to read reviews, get an understanding of what the customer is trying to convey about the company, and use the information to make a differentiated rating.
\end{abstract}

\section{Introduction}
When reading reviews at Trustpilot, you often see reviews which are very one-sided, only talking about how good a store is at handling delivery, how good their prices are or how good their presales support is. The problem with Trustpilot's rating system is, that it is merely the average of all ratings for the company. As users do not touch on all subjects the system is inherintly flawed, as a user who only touches one subject is rated the same as a user who touches all.

An example would be a customer writing a review, which only has information about how quick the company is to deliver a product. It is reasonable to presume, that the rating does not tell anything, about how good the company is at handling RMAs (Return Merchandize Authorization). 

By reading reviews and classifying the rating as something that is only related to delivery time, it is possible to make a differentiated rating, which this project achieves by using topic classification and sentiment analysis.

\section{Methods}
As the purpose of this project is to improve the rating system of a website, it makes sense to extend its functionality. This is achieved by using a \texttt{Greasemonkey}\cite{GreaseMonkey} script and an accompanying website which the script can acquire data from using \href{http://en.wikipedia.org/wiki/Ajax_(programming)}{\texttt{Ajax}}.
The system requires a web server, for this we use \href{http://httpd.apache.org/}{\texttt{Apache HTTP Server}}, as it is relatively easy to setup when using Linux. As the system needs to store data such as the differentiated ratings, a database is necessary, for which we use \href{http://www.mysql.com/}{\texttt{MySQL}}.

As the project is part of the course \href{http://www.kurser.dtu.dk/02819.aspx?menulanguage=en-GB}{02819 Data Mining using Python}, the project is coded in Python and uses language specific libraries.

\section{Results}


\subsection{Code checking}


\subsection{Testing}


\subsection{Profiling}


\section{Discussion}


\section{Conclusion}

\bibliographystyle{IEEEtran}
\bibliography{References}

\end{document}
